\documentclass[landscape,10pt]{article}
%\documentclass[a4paper,landscape,10pt]{article}
\usepackage{amsmath,pslatex,amssymb,amsfonts,latexsym,amstext,amsmath,amsthm,enumerate,color,listings,multicol}
\usepackage[usenames,dvipsnames]{xcolor}
\usepackage{fullpage}
\usepackage{fancyhdr}

\usepackage{setspace}
\usepackage{lstlinebgrd}
\usepackage{calc}

\usepackage{graphicx}
\DeclareGraphicsExtensions{.pdf,.png,.jpg,.svg}

\setcounter{tocdepth}{2}
%\parskip .2cm
\renewcommand{\contentsname}{\vspace{-\baselineskip}}
\newcounter{lnum} 
\pagestyle{fancy}
\headheight 35pt
\rhead{\textbf{\thepage}}
\lhead{\textbf{University of Lethbridge}}
\cfoot{}
\setlength{\voffset}{-0.8in}
\setlength{\hoffset}{-0.77in}
\setlength{\headsep}{0.1in}
\addtolength{\textheight}{1.25in}
\addtolength{\textwidth}{1.15in}
\setlength{\columnseprule}{1pt}
\setlength{\marginparwidth}{20pt}
\definecolor{light-gray}{gray}{0.90}
\lstset{
  language=C++,
  basicstyle=\scriptsize\ttfamily\bfseries,
  numbers=none, 
  numberstyle=\tiny, 
%  stepnumber=5,
  linebackgroundcolor={\setcounter{lnum}{\value{lstnumber} / 5} \ifodd \value{lnum} \color{light-gray}\fi},
  linebackgroundsep={0.11in},
  keywordstyle=\color{blue}, 
  commentstyle=\color{red},
  stringstyle=\ttfamily, % typewriter type for strings
  frame=bt, 
  rulesepcolor=\color{red!20!green!20!blue!20},
  rulecolor=\color{Brown},
  escapeinside=``, 
  %xleftmargin=2em,xrightmargin=2em, aboveskip=1em,
  breaklines=true             % sets automatic line breaking
}

\title{ACM ICPC World Finals 2014 Code Booklet}
\author{University of Lethbridge}
\date{}

\newcommand{\code}[2]{\addcontentsline{toc}{subsection}{#2}\lstinputlisting{#1}}
\newcommand{\directed}{$^{*}$}
\newcommand{\undirected}{$^{**}$}
\newcommand{\both}{$^{***}$}
\newcommand{\myimage}[2]{\begin{center}\includegraphics[height=#2]{Images/#1}\end{center}}

\begin{document}

\begin{center}
 {\Huge\bf{ACM ICPC World Finals 2020 Code Booklet}}\\\vskip5pt
 {\Huge University of Lethbridge}
\end{center}
%\maketitle
\thispagestyle{fancy}


\begin{multicols*}{2}
  \singlespacing
  \small
  \tableofcontents

\section{Setup}
  \code{run}{Automated compile and test}
  \code{template.cc}{c++ template file}
  \code{vimrc}{vim settings}

\section{Geometry}
  \code{geometry2D.cc}{2D Geometry routines}
  \code{geometry3D.cc}{3D Geometry routines}
  \code{delauney_triangulation.cc}{Delauney Triangulation}
  \code{greatcircle.cc}{Great Circle Distance}
  \code{heron.cc}{Heron's Triangle Formula}
  \code{union_rectangles.cc}{Area of Union of Rectangles}
  \code{IntegerGeometry.cc}{Integer 2D Geometry}

\section{Math}
  \code{algebra.cc}{Polynomial Algebra}
  \code{matrix.cc}{Square Matrix Exponentiation}

\subsection{Number Theory}
  \code{cra.cc}{Chinese Remainder Theorem}
  \code{euclid.cc}{Extended Euclidean}
  \code{discrete_log.cc}{Discrete Log Solver}
  \code{ntt.cc}{Number Theory Convolution}
  %\code{factor_large.cc}{Large Primality Testing / Factor Large Numbers}

\subsection{Linear Algebra}
  \code{diophantinesolve.cc}{System of Linear Diophantine Equations}
  \code{fflinsolve.cc}{Fraction Free Linear Solve}
  \code{linsolve.cc}{LU-Decomposition}

%\section{Big Integer}
  %\code{bigint.cc}{Big Integers}
  %\code{int128.cc}{128-Bit Integers}

\section{Dynamic Programming}
  \code{lis.cc}{Longest Increasing Subsequence}
  \code{vecsum.cc}{Vector Sum}
  %\code{manacher.cc}{Longest Palindrome Substring}

\section[Graph Theory ~~ (Directed\directed ~~ or ~~ Undirected\undirected ~~ or  ~~ Both\both)]{Graph Theory}
  \code{graph_layout.cc}{General Graph Layout}
  \code{bellmanford.cc}{Bellman-Ford Algorithm \both}
  \code{eulertour.cc}{Eulerian Tour \both}
  \code{genmatching.cc}{General Graph Matching \undirected}
  \code{mincostmaxflowsparse.cc}{Min-Cost Max-Flow for Sparse Graphs \both}
  \code{min_cut.cc}{Minimum Cut of a Graph \undirected}
  \code{networkflow.cc}{Network Flow $O(fm)$ \both}
  \code{networkflow2.cc}{Network Flow $O(n^3)$ \both}

  % don't require graph-layout
  %\code{dinic.cc}{Dinic's Max Flow}
  \code{hungarian.cc}{Hungarian Algorithm}
  \code{matching.cc}{Bipartite Matching}
  \code{SCC.cc}{Strongly Connected Components}
  \code{2SAT.cc}{2SAT}

\section{Data Structures}
  \code{fenwick.cc}{Fenwick Tree}
  \code{segment1.cc}{Segment Tree}
  \code{lazysegment.cc}{Lazt Segment Tree}
  \code{unionfind.cc}{Union Find}
  \code{CHT.cc}{Dynamic Hull (CHT)}
  \code{CentroidTree.cc}{Centroid Decomposition}
  \code{min_queue.cc}{Queue with Minimum Maintained}
  %\code{segment2.cc}{Segment Tree}
  %\code{fraction.cc}{Fraction Class}
  %\code{date.cc}{Date Class}
  %\code{priority_queue.cc}{Priority Queue with Handles}

\section{String Processing}
  \code{kmp.cc}{KMP}
  \code{trie.cc}{Trie datastructure}
  \code{SA.cc}{Suffix Array}
  \code{str_rotation_period.cc}{String Rotation Period}

\section{Algorithms and Misc}
  \code{alpha_beta.cc}{Alpha-Beta pruning}
  \code{simpsons_rule.cc}{Simpson's Rule}
  \code{simplex.cc}{Simplex Algorithm}
  \code{poly_mult.cc}{FFT (floating point)}
  \code{hashing}{Primes for hashing}

\section{Formulas}

%\subsubsection*{Because Darcy is Dumb...}

%$v = \frac{d}{t}$ \hskip20pt $d = vt$ \hskip20pt $t = \frac{d}{v}$

\subsubsection*{Triangles}

\begin{description}
\item[Sine law:] $\frac{\sin(\alpha)}{a} = \frac{\sin(\beta)}{b} =
  \frac{\sin(\gamma)}{c}$, $a,b,c$ = side lengths, $\alpha, \beta,
  \gamma$ = opposite angles.
\item[Cosine law:] $c^2 = a^2 + b^2 - 2ab\cos(\gamma)$
\item[Circle inscribed in triangle:] radius = $\sqrt{\frac{(s-a)(s-b)(s-c)}{s}}$,
  $s = \frac{a+b+c}{2}$.
\item[Circumcircle:] radius = $\frac{abc}{4A}$, $A$ = area of triangle.
\end{description}

\subsubsection*{Trig Identities}

\begin{tabbing}
$\sin^2(u)$ \hskip30pt\=$= \frac12 (1-\cos(2u))$ \hskip70pt \=
$\cos^2(u)$ \hskip30pt\=$= \frac12 (1+\cos(2u))$ \\
$\sin(u) + \sin(v) $ \> $= 2\sin\left(\frac{u + v}{2}\right)\cos\left(\frac{u - v}{2}\right)$ \>
$\sin(u) - \sin(v) $ \> $= 2\sin\left(\frac{u - v}{2}\right)\cos\left(\frac{u + v}{2}\right)$\\
$\cos(u) + \cos(v) $ \> $= 2\cos\left(\frac{u + v}{2}\right)\cos\left(\frac{u - v}{2}\right)$ \>
$\cos(u) - \cos(v) $ \> $= -2\sin\left(\frac{u + v}{2}\right)\sin\left(\frac{u - v}{2}\right)$ \\
$\sin(u)\sin(v) $ \> $= \frac12\left(\cos(u-v) - \cos(u+v)\right)$ \>
$\cos(u)\cos(v) $ \> $= \frac12\left(\cos(u-v) + \cos(u+v)\right)$\\
$\sin(u)\cos(v) $ \> $= \frac12\left(\sin(u+v) + \cos(u-v)\right)$ \>
$\cos(u)\sin(v) $ \> $= \frac12\left(\sin(u+v) - \cos(u-v)\right)$
\end{tabbing}

\myimage{TrigCircle.pdf}{1.5in}

{\bf Length of a Chord:} $2r\sin\theta$

\subsubsection*{Other Geometry}

\begin{description}
\item[Rotation matrix:] 
  $\begin{bmatrix} \cos \theta & -\sin \theta \\
  \sin \theta & \cos \theta \end{bmatrix}$ (counter-clockwise by $\theta$)
\item[Dot product:] $\vec u \cdot \vec v = \|\vec u\| \|\vec v\| \cos \theta$.
\item[Sphere through 4 Points:] Given $(x_i,y_i,z_i)$, find $(x,y,z)$ and $r$.

$x = 0.5 \cdot M_{12}/M_{11}$, $y = -0.5 \cdot M_{13}/M_{11}$, $z = 0.5 \cdot M_{14}/M_{11}$, $r = d((x,y,z),(x_1,y_1,z_1))$

where
$\left|\begin{array}{ccccc}
x^2+y^2+z^2 & x & y & z & 1 \\
x_1^2+y_1^2+z_1^2 & x_1 & y_1 & z_1 & 1 \\
x_2^2+y_2^2+z_2^2 & x_2 & y_2 & z_2 & 1 \\
x_3^2+y_3^2+z_3^2 & x_3 & y_3 & z_3 & 1 \\
x_4^2+y_4^2+z_4^2 & x_4 & y_4 & z_4 & 1
\end{array}\right| = 0$
\end{description}

\subsubsection*{Number Theory}

\begin{description}
\item[Number and sum of divisors:] multiplicative, $\tau(p^k) = k+1$,
 $\sigma(p^k) = \frac{p^{k+1}-1}{p-1}$.
\item[Linear Diophantine equations:] $a\cdot s + b\cdot t = c$ iff
  $\gcd(a,b) | c$. 

Solutions are $(s_0, t_0) + k \cdot \left(\frac{b}{\gcd(a,b)}, -\frac{a}{\gcd(a,b)}\right)$.
\end{description}

\subsubsection*{Misc}

\begin{description}
\item[Pick's Theorem:] $A = i + \frac{b}{2} - 1$, $A$ = area, $i$ = interior
  lattice points, $b$ = boundary lattice points.
\item[Euler formula:] $V - E + F - C = 1$, $V$ = vertices, $E$ =
  edges, $F$ = faces, $C$ = number of connected components.  True for
  planar graphs and regular polyhedra (assume $C = 1$ in the latter).
\item[Catalan numbers:] $C_n = \frac{1}{n+1} \binom{2n}{n}$.
  Recurrence: $C_0 = 1$, and $C_{n+1} = \sum_{i=0} C_i C_{n-i}$.
\item[Derangements:] $!0 = 1$, $!1 = 0$, $!n = (n-1)(!(n-1) + !(n-2))$.
\item[Burnside's Lemma:] $|X/G| = \frac{1}{|G|}\sum_{g\in G} |X_g|$ (Points fixed by $g$) $[\frac{1}{24}(n^6+3n^4+12n^3+8n^2)]$
\item[Number of solutions:] $x_1 + \cdots + x_k = r$ with $x_i \geq 0$:
  $\binom{r+k-1}{r}$
\item[Integer Partitions of $n$:] (Also number of nonnegative solutions to $b+2c+3d+4e+...=n$ and the number of nonnegative solutions to $2c+3d+4e+...\leq n$)\\

\hskip-20pt
{\footnotesize
\begin{tabular}{|r|rrrrrrrrrr|}
\hline
   & x0 & x1 & x2 & x3 & x4 & x5 & x6 & x7 & x8 & x9 \\ \hline
0x & 1 & 1& 2& 3& 5& 7& 11& 15& 22& 30 \\
1x & 42& 56& 77& 101& 135& 176& 231& 297& 385& 490 \\
2x & 627& 792& 1002& 1255& 1575& 1958& 2436& 3010& 3718& 4565 \\
3x & 5604& 6842& 8349& 10143& 12310& 14883& 17977& 21637& 26015& 31185 \\
4x & 37338& 44583& 53174& 63261& 75175& 89134& 105558& 124754& 147273& 179525 \\\hline
\end{tabular}
}

\item[Lagrange Interpolation:] Given $(x_0,y_0), \cdots , (x_n,y_n)$, the polynomial is:

$P(x) = \displaystyle\sum_{j=1}^{n}P_j(x)$ where $P_j(x) = y_j\displaystyle\prod_{0 \leq k \leq n, k \neq j}\frac{x-x_k}{x_j-x_k}$

\item[Usable Chooses:] $\binom{n}{k}$ is safe assuming 50,000,000 is not TLE: $\binom{28}{k}$ is okay for all $k \leq n$.

\hskip-20pt
$
\begin{array}{|c|ccccccccc|}
 \hline
 n & 29 & 30-31 & 32-33 & 34-38 & 39-45 & 46-59 & 60-92 & 93-187 & 188-670 \\ \hline
 k & 11 & 10 & 9 & 8 & 7 & 6 & 5 & 4 & 3 \\\hline
\end{array}
$
\end{description}

\subsubsection{Physics}

\myimage{Ellipse.jpg}{1.25in}

\begin{description}
 \item[Circumference:] $4a\int_0^{\pi/2}\sqrt{1-\epsilon^2\sin^2(\theta)} d\theta$
 \item[Polar form relative to focus:] $r(\theta) = \frac{a(1-\epsilon)}{1-\epsilon\cos(\theta-\phi)}$ , where $\phi$ is the angle of rotation of ellipse.
 \item[Polar form relative to centre:] $r(\theta) = \frac{ab}{\sqrt{(b\cos\theta)^2+(a\sin\theta)^2}}$
 \item[Minimal Surface of Revolution (Rotating around x-axis):] $y=a\cosh(\frac{x-b}{a})$ \\ Do binary search on $a$ using secant lines -- $(a,b)$ is the extrema
\item[Rational Roots:] $a_nx^n + \cdots + a_0 = 0$. If $\frac{p}{q}$ is a solution, where $(p,q)=1$, then $p | a_0$ and $q | a_n$.
\end{description}

$r^2\frac{d\theta}{dt} = \frac{2\pi}{p} ab$

\subsection{Rotating Calipers}

\begin{description}
\item[Computing distances:]
The diameter of a convex polygon, The width of a convex polygon, The maximum distance between 2 convex polygons, The minimum distance between 2 convex polygons.
\item[Enclosing rectangles:]
The minimum area enclosing rectangle, The minimum perimeter enclosing rectangle
\item[Triangulations:]
Onion triangulations, Spiral triangulations Quadrangulations
\item[Properties of convex polygons:]
Merging convex hulls, Finding common tangents, Intersecting convex polygons, Critical support lines, Vector sums of convex polygons
\item[Thinnest transversals:]
Thinnest-strip transversals
\end{description}

\section{Tips}

\subsubsection*{If You Are Stuck, Read These!}

\begin{itemize}
 \item Can you write the question as a whole bunch of inequalities? (Simplex?)
 \item Can you hash to reduce time? (Normally cuts a factor of N)
 \item Can you \underline{only} have one ``item'' on a location at a time? Can \underline{only} one ``item'' move through a hallway at one time?
 \item Can you break the problem into two disjoint sets? (Even/Odd, Black/White, 2-player games)
 \item Is $n \approx 40$? Consider $O(2^{n/2}\log(2^{n/2}))$.
 \item Would $\sqrt{N}$ blocks of size $\sqrt{N}$ help?
 \item Read the Table of Contents!
 \item Binary search and check (often greedy)
 \item Sweep line/circle (often with extra data structures)
 \item DP:
   \begin{itemize}
   \item subsets (e.g. TSP type)
   \item on trees: state = (root, extra info)
   \item on DAG
   \item incremental convex hull/envelope code
   \item probability/expected value in a state transition graphs, deal
     with cycles through infinite series or linear equations.
   \end{itemize}
 \item Represent moving objects as $f(t) = v \cdot t +
   \text{init. pos}$. and use geometry.
 \item Coordinate compression
 \item Meet-in-the-middle
 \item Max flow of some kind, but need to formulate right graph
 \item Brute Force:
   \begin{itemize}
   \item Are there very few different solutions?
   \item Are there very few different (effective) inputs?
   \item Pruning
   \end{itemize}
 \item Math:
   \begin{itemize}
   \item integration/area computation
   \item physics: make sure you read all the rules
   \end{itemize}
 \item Game Theory (2-player):
  \begin{itemize}
   \item Can you duplicate your opponent's move?
   \item Can formulate it so one person is maximizing something and one person minimizing?
   \item Write a program to brute force small cases and look for a pattern.
   \end{itemize}
 \item Try to looking at the problem in reverse?
 \item Cycle decomposition of permutation.
\end{itemize}

\subsubsection*{General Things}

\begin{itemize}
 \item {\color{red} \textbf{RTFQ} }
 \item Step away from the computer.  Go to the bathroom.
 \item Print after every submission, debug on paper.
 \item Did you remember to handle the empty cases (e.g. n = 0).
 \item Graphs: is it directed or undirected?
 \item Floating-point computation: be careful about -0.0
 \item atan2 can return -pi and +pi
 \item Watchout for stack overflow (DFS and large variables)
\end{itemize}

%\subsubsection*{Random Images}

\subsubsection*{Point and Circle Tangent}
\myimage{point_tangent.png}{1.25in}
Now Intersect two circles: $(C,R)$ and $(P,L=\sqrt{D^2-R^2})$.

\myimage{cir_cir_tang.png}{1.25in}
Two circles of radii $r_1 \leq r_2$. For {\bf outer tangent} (Left Picture), make a circle of radius $r_2-r_1$ around $C_2$ (dashed circle) and find tangent lines from $C_1$ ({\color{blue} dashed blue line}), then translate it $r_1$ units ({\color{blue} solid blue line}). For {\bf inner tangent} (Right Picture), make a circle of radius $r_1+r_2$ around $C_1$ (dashed circle) and find tangent lines from $C_2$ ({\color{red} dashed red line}), then translate it $r_2$ units ({\color{red} solid red line}).

\subsubsection*{Holomorphic Centre}
\myimage{holo_centre.pdf}{1.5in}
Inner tangent lines go through $I$: \\
$I = (x,y) = \frac{r_2}{r_1+r_2}(x_1,y_1) + \frac{r_1}{r_1+r_2}(x_2,y_2)$ \hskip 20pt
$E = (x,y) = \frac{-r_2}{r_1-r_2}(x_1,y_1) + \frac{r_1}{r_1-r_2}(x_2,y_2)$

\subsubsection*{Power Points}
\myimage{power_point.pdf}{1.5in}
$\overline{PT}^2 = \overline{PM} \cdot \overline{PN} = \overline{PA} \cdot \overline{PB} = \overline{PO}^2 - \overline{TO}^2$

\subsubsection*{Start of Contest}

\begin{itemize}
 \item Put this somewhere in the \texttt{.bashrc} file:
   \begin{lstlisting}
    function amake(){
      g++ -g -std=gnu++0x -static -Wall ${1}.cc -o ${1}
    }
    ulimit -c unlimited
    function e { emacs "$@" & }
   \end{lstlisting}
 \item Type this command: \texttt{source .bashrc}
\end{itemize}

\myimage{Inspiration.jpg}{1.3in}

\end{multicols*}
\end{document}
